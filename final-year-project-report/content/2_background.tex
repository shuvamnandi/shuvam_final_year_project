\chapter{Background}
\label{ch2_background}

Basics of certain programming knowledge and possessing concepts along with the ability to use some softwares, tools and devices is a necessity to handle the tasks in this project. OpenCL has been used extensively throughout the length of this project, which requires thorough understanding of how parallel programs are executed on heterogeneous devices. The notion of host and device must be clear in order to conceptualise the execution of programs on the device.\newline\newline
A variety of co-processors like GPUs, FPGAs are investigated for performance metrics in this project. Therefore, familiarity with these computing devices is an added benefit. A high-level understanding of neural networks proved favourable while running profiling experiments on a CNN application. In order to evaluate performance, metrics such as Operations per second must be known.\newline\newline
The project also discusses the details of RISC-V ISA, requiring the foundational knowledge of computer organisation and architecture.  This is succeeded by implementing a new processor based on the ISA using Verilog. In order to evaluate the functional correctness of the implementation, comprehension of Register Transfer Level (RTL) simulation is a requisite.  Icarus Verilog is used to run the testbench on Linux while ModelSim SE is used on Windows platform.\newline\newline
GNU utilities are used in this project for compiling programs written in C and OpenCL. All work is done on a PC running Ubuntu 14.04 and Windows 10 in dual-boot mode.