\chapter{Background}
\label{ch2_background}

Basics of certain programming knowledges and possessing concepts along with the ability to use some softwares, tools and devices is a necessity to handle the tasks in this project. OpenCL has been used extensively throughout the length of this project, which requires thorough understanding of how programs are executed on heterogeneous devices. A basic knowledge of how parallel computing improves performance is key to understand why acceleration happens. The notion of host and device must be clear in order to conceptualise the execution of programs on the device.\newline\newline
A variety of co-processors like GPUs, FPGAs are investigated for performance metrics in this project. Therefore, familiarity with these computing devices is an added benefit. An application implementing the LeNet-5 architecture of convolutional neural networks is profiled in this project. A high-level understanding of neural networks in general proved favourable. In order to evaluate performance, metrics such as Instructions per second and Operations per second must be known.\newline\newline
The project also discusses the details of RISC-V ISA, which requires the foundational knowledge of computer organisation and architecture.  This is succeeded by implementing a new processor based on the ISA, requiring practice in describing hardware using Verilog. In order to successfully evaluate the functional correctness of the implementation, comprehension of Register Transfer Level (RTL) simulation is a requisite.  Icarus Verilog is used to run the testbench on Linux while ModelSim SE is used on Windows platform for the same.\newline\newline
GNU utilities are used extensively in this project for compiling and debugging programs written in C and OpenCL. All work done was on a PC running Ubuntu 14.04 and Windows 10 on dual-boot mode.