\chapter*{Abstract} 
\label{ch0i_Abstract}

%\begin{abstract}
%With the advancements in technology, parallel processing architectures such as multi-core processors, digital signal processors (DSPs), graphics processing units (GPUs), massively parallel processor arrays (MPPAs) and field programmable gate array (FPGA) based accelerators are gaining popularity for accelerated execution of compute kernels.
%Research efforts have shown strength of FPGA accelerators in a wide range of application domains where compute kernels can be implemented as high performance fully parallel and pipelined designs.
%Despite these advantages, FPGAs have not yet been ready for mainstream computing.
%One reason is that design productivity remains a major challenge, restricting the effective use of FPGA accelerators to niche disciplines involving highly skilled hardware engineers.
%Coarse-grained FPGA overlay architectures have been shown to be effective when paired with general purpose processors, offering software-like programmability, fast compilation, application portability and improved design productivity.
%These architectures enable general purpose hardware accelerators, allowing hardware design at a higher level of abstraction.
%This report presents a placement and routing (PAR) tool for coarse grained island-style overlays based on the algorithms used in widely accepted VPR placement and routing tool.
%We start with understanding the PAR algorithms in detail and develop a python based PAR tool customized for coarse-grained island-style overlays.
%We aim to adapt the algorithms to support different interconnect architectures as a future work.
 
%\end{abstract}