\chapter{Introduction}
\label{ch1_introduction}
\section{Motivation}
Silicon technology will continue to provide an exponential increase in the availability of raw transistors. 
Effectively translating this resource into application performance, however, is an open challenge that conventional processor designs will not be able to meet. 
On the other hand, Field Programmable Gate Array (FPGA) devices provide a sea of high performance computing blocks for implementing kernels as high performance fully parallel and pipelined designs.
For more than a decade, researchers have shown that FPGAs can accelerate a wide variety of software, in some cases by several orders of magnitude compared to state-of-the-art general purpose processors.
The most fundamental difference is that general-purpose processors provide functionality to execute a list of instructions sequentially, whereas FPGA architectures can implement compute kernels by mapping compute operations on configurable blocks.

While the performance benefits of FPGAs over processor based systems have been well established~\cite{compton_reconfig_2002,tessier2015reconfigurable,dehon2015fundamental,trimberger2015three}, such platforms have not seen wide use beyond specialist application domains such as digital signal processing and communications.
Poor design productivity has been a key limiting factor, restricting their effective use to experts in hardware design~\cite{stitt2011field}. 
Even as \ac{HLS} tools improve in efficiency~\cite{canis_legup:_2011,liang_high-level_2012}, prohibitive compilation time (specifically place and route time) still limits productivity and mainstream adoption of reconfigurable platforms.

Despite numerous efforts in reducing reconfiguration times and improving CAD tool support for dynamic reconfiguration of FPGA fabric~\cite{vipin2012,vipin_high_2012,vipin2013,vipin_automated_2014,vipin_zycap_2014}, It still prevents designers from using FPGA as a rapidly reconfigurable hardware accelerator.
Thus, the requirement to rapidly change the hardware fabric, that is to perform a hardware context switch, has led to the development of coarse grained overlay architectures which allow for fast compilation and software like
programmability.


Coarse-grained FPGA overlay architectures~\cite{plessl_zippy_2005,bergmann_quku:_2013,coole_intermediate_2010,capalija_high-performance_2013,govindaraju2012dyser,fccm2015-jain,heart2015-jain,liu_soft_2013,cong2014fully} have been shown to be effective when paired with general purpose processors, offering software-like programmability, fast compilation, application portability and improved design productivity.
These architectures enable general purpose hardware accelerators, allowing hardware design at a higher level of abstraction.
In our work, we aim to develop a placement and routing (PAR) tool for coarse grained island-style overlays.
%Eventually we aim to adapt the algorithms to support different interconnect architectures.


\section{Contribution}


\section{Organization}
The remainder of the report is organized as follows: 
Chapter \ref{ch2_background} presents background information on
Chapter \ref{ch3_lit_review} studies current state of the art overlays and techniques for placement and routing.
Chapter \ref{ch4_opencl} 
Chapter \ref{ch5_cnn} throws light 
Chapter \ref{ch6_riscv} shows 
We thereby conclude in chapter \ref{ch7_conclusion} and discuss future work.
